\documentclass[12pt]{article}

\usepackage{amsmath}
\usepackage{amsthm}
\usepackage[margin=2.5cm]{geometry}
\usepackage{amsfonts}
\usepackage{amssymb}

\newcommand{\N}{\mathbb{N}}
\newcommand{\Z}{\mathbb{Z}}
\newcommand{\Q}{\mathbb{Q}}
\newcommand{\R}{\mathbb{R}}
\newcommand{\C}{\mathbb{C}}

\def\plus{\text{+}}

% Document starts here

\begin{document}

\pagenumbering{gobble}

\setlength\parindent{0pt}

\section*{Question 1}

\textbf{Required:} Prove that if $P = NP$, then $EXP = NEXP$ 

\begin{proof}
Assume $P = NP$. \\

Trivially, we know that $EXP \subseteq NEXP$ since any deterministic exp-time TM can be viewed as a nondeterministic exp-time TM (without any non-trivial branching from the transition function). So any langauge decided by a deterministic exp-time TM can be decided by a nondeterministic exp-time TM. \\

Hence, we just need to show $NEXP \subseteq EXP$. Let $A \in NEXP$. Hence, there is a nondeterministic exp-time $TM$ $M$ that decides $A$ in $\mathcal{O}(2^{n^k})$ where $n$ is the input length. \\

We want to show $A \in EXP$. As the hint suggests, let $A^\prime = \{x1^{2^{|x|^k}} : x \in A\}$ be the corresponding padded language. \\

First we will show that $A^\prime \in NP$. Consider the following nondeterministic polytime $TM$ $N_1$ that decides $A^\prime$. \\

$N_1 = $ "on input $w$: \\

\setlength\parindent{15pt}

\indent 1. Test that $w =x1^{2^{|x|^k}}$ for some string $x$. If not, then reject immediately. \\

\indent 2. Run $M$ on input $x$ nondeterministically. If $M$ accepts on any branch, then accept. Otherwise, reject." \\

\setlength\parindent{0pt}

Clearly $N_1$ is a decider for $A^\prime$ since $M$ is a decider for $A$. Clearly Step 1 takes polynomial time with respect to the input length. Notice that on an input $w = x1^{2^{|x|^k}}$, on Step 2, $M$ runs on $x$ which takes time $\mathcal{O}(2^{|x|^k})$. Hence, the overall runtime of $N_1$ is polynomial with respect to the original input length $|x1^{2^{|x|^k}}|$. Hence, $N_1$ is a poly-time nondeterministic TM that decides $A^\prime$. \\

Hence, $A^\prime \in NP$. Since we assumed $P = NP$, we have that $A^\prime \in P$. \\

Since $A^\prime \in P$, there exists some poly-time deterministic TM $N_2$ that decides $A^\prime$. \\

Now we will define an exp-time deterministic TM $H$ that decides $A$. \\

$H = $ "on input $x$: \\

\setlength\parindent{15pt}

\indent 1. Let $w = x1^{2^{|x|^k}}$. i.e. $w$ is a padding of $x$. \\

\indent 2. Run $N_2$ on $w$. If $N_2$ accepts, then accept. If $N_2$ rejects, then reject." \\
\setlength\parindent{0pt}

Clearly $H$ is a decider for $A$ since $N_2$ is a decider for $A^\prime$. Step 1 takes exponential time with respect to the input length $|x|$ since Step 1 is an exponential padding of $x$. On Step 2, $N_2$ will run on $w = x1^{2^{|x|^k}}$. Since $N_2$ is polytime, we have that $N_2$ will run in polynomial time with respect to $|x1^{2^{|x|^k}}|$. But $|x1^{2^{|x|^k}}|$ is exponential with respect to the original input length $|x|$. Hence, $H$ will run in exponential time in terms of $|x|$. Hence, $H$ is an exp-time deterministic $TM$ that decides $A$. Hence, $A \in EXP$. \\

Therefore, $NEXP \subseteq EXP$. \\

Therefore, we have shown $EXP = NEXP$, as required. 
\end{proof}


\newpage

\section*{Question 2}

\textbf{Required:} Show that $MINFORMULA \in PSPACE$. \\

Consider the following TM $M$ that decides $MINFORMULA$ in polynomial space. \\

$M = $ "on input $w$: \\

\setlength\parindent{15pt}

\indent \textbf{1.} Check that $w = \langle \phi \rangle$ for some formula $\phi$. If not, reject immediately. \\

\textbf{2.} Otherwise, run the following loop. \\

For each string $s$ such that $|s| < |\langle \phi \rangle|$, execute the following substeps i) and ii): \\

i) Check that $s = \langle \psi \rangle$ for some formula $\psi$ such that $\phi$ and $\psi$ have the same set of variables. If not, move on to the next iteration of the for loop (if there is a new iteration). \\

ii) Evaluate $\phi$ and $\psi$ on every possible truth assignment. If $\phi$ and $\psi$ evaluate to the same truth value on every truth assignment, then reject. \\

\textbf{3.} If we have not entered the reject state after Step 2, then accept." \\

\setlength\parindent{0pt}

Clearly $M$ decides $MINFORMULA$ since if $w \in MINFORMULA$, then $w = \langle \phi \rangle$ for some formula $\phi$. And since $\phi$ is minimal, there is no formula $\psi$ with the same variables and is of shorter length that is equivalent to $\phi$. Hence, we cannot possibly ever reject on Step 2 ii). Hence, we will accept on Step 3. \\

And if $w \not \in MINFORMULA$, then either $w$ is not an encoding of a formula and we will reject immediately on Step 1; or $w = \langle \phi \rangle$ such that there is some formula $\psi$ with the same variables where $|\langle \psi \rangle| < |\langle \phi \rangle|$ and $\psi$ is equivalent to $\phi$. Our TM $M$ will eventually find $\langle \psi \rangle$, and we will reject on an iteration of Step 2 ii). \\

Hence, $M$ is a decider for $MINFORMULA$. \\

To show that $M$ runs in polynomial space, notice that in Step 2, we are only considering strings $s$ such that $|s| < |\langle \phi \rangle| = |w|$. So we have $\mathcal{O}(|w|)$ space here. And we can reuse this space for each string $s$ that is considered. Similarly, for any $s = \langle \psi \rangle$, any truth assignment for $\phi$ and $\psi$ will take $\mathcal{O}(m)$ space where $m$ is the number of variables in both $\phi$ and $\psi$. We can reuse this space for each assignment that we are considering. And since the number of variables $m$ is at most $|w| = |\langle \phi \rangle|$, we have $\mathcal{O}(|w|)$ space. So $M$ runs in $\mathcal{O}(|w|)$ space. \\

Therefore, $MINFORMULA \in PSPACE$, as required.  

\newpage

\section*{Question 3}

\textbf{Required:} Prove that if $P = NP$, then $MINFORMULA \in P$. 

\begin{proof}
Assume $P = NP$. \\

First we will show that the following language $L$ is such that $L \in NP$. \\

$L = \{\langle \phi, \psi \rangle : \text{$\phi$ and $\psi$ have the same variables and $\phi$ is not equivalent to $\psi$}\}$ \\

Consider the following verifier $V_1$ for $L$ which uses a truth assignment that makes one of $\phi$ and $\psi$ true, and the other false as a certificate. \\

$V_1 = $ "on input $\langle \langle \phi, \psi \rangle, c \rangle$: \\

\setlength\parindent{15pt}

\indent 1. Test that $\phi$ and $\psi$ have the same variables. \\

\indent 2. Test that $c$ is an encoding of a truth assignment to the variables in $\phi$ and $\psi$. \\

\indent 3. Test that one of $\phi$ and $\psi$ is true under the assignment encoded by $c$, and the other is false. \\

\indent 4. If all tests pass, then accept. Otherwise reject." \\

\setlength\parindent{0pt}

Clearly $V_1$ is a verifier for $L$ since if $w = \langle \phi, \psi \rangle \in L$, then $\phi$ and $\psi$ have the same variables and are not equivalent, and hence there is a truth assignment that makes one true and the other false. This truth assignment can be encoded in a certificate $c$. And if $w \not \in L$, then either $w$ is not of the correct form, or $w = \langle \phi, \psi \rangle$ and $\phi$ and $\psi$ are equivalent. Hence, there can be no valid certificate $c$. \\

To show that $V_1$ is a poly-time verifier, notice that any truth assignment just has to assign truth values to the variables in $\phi$ and $\psi$. The number of variables is less than $|\langle \phi, \psi \rangle|$. Hence, $|c| < |\langle \phi, \psi \rangle|$. And each test clearly takes polynomial steps with respect to the input length. Hence, $V_1$ is a poly-time verifier.  \\

Hence, $L \in NP$. Since we assumed $P = NP$, we have that $L \in P$. Hence, there is a TM $M$ that decides $L$ in polynomial time. \\

Now consider the following verifier $V_2$ for $\overline{MINFORMULA}$. \\

$V_2 = $ "on input $\langle \phi, c \rangle$: \\

\setlength\parindent{15pt}

\indent 1. Test that $c = \langle \psi \rangle$ for some formula $\psi$ such that $|\langle \psi \rangle| < |\langle \phi \rangle|$ and $\phi$ and $\psi$ share the same set of variables. If not reject, immediately. \\

\indent 2. If Test 1 passes, run $M$ on input $\langle \phi, \psi \rangle$. If $M$ rejects, then accept. If $M$ accepts, then reject. i.e. Do the opposite of what $M$ does." \\
\setlength\parindent{0pt}

Clearly $V_2$ is a verifier for $\overline{MINFORMULA}$ since if $w \in \overline{MINFORMULA}$, then either $w$ is not an encoding of a formula, or $w = \langle \phi \rangle$ such that $\phi$ is not minimal. Hence, there is some formula $\psi$ with the same set of variables as $\phi$ such that $|\langle \psi \rangle| < |\langle \phi \rangle|$ and $\phi$ is equivalent to $\psi$. Hence, $c = \langle \psi \rangle$ is a certificate. And if $w \not \in \overline{MINFORMULA}$, then $w = \langle \phi \rangle$ for some minimal formula $\phi$. Hence, we cannot have a valid certificate $c$. \\

To show that $V_2$ is a poly-time verifier, first notice that an accepting certificate $c$ is such that $c = |\langle \psi \rangle| < |\langle \phi \rangle|$. Clearly Step 1 takes polynomial time with respect to the length of the input. In Step 2, $M$ is polynomial with respect to its input length $|\langle \phi, \psi \rangle|$. But since $|\langle \psi \rangle| < |\langle \phi \rangle|$, we know that $|\langle \phi, \psi \rangle| \leq k|\langle \phi \rangle|$ for some constant $k$. This constant $k$ would depend on the specifics of our coding scheme. Regardless, we have that $M$ runs in polynomial time with respect to $|\langle \phi \rangle|$. Hence, our verifier $V_2$ runs in polynomial time with respect to $|\langle \phi \rangle|$. \\

Hence, $V_2$ is a poly-time verifier for $\overline{MINFORMULA}$. Hence, $\overline{MINFORMULA} \in NP$. \\

Since $\overline{MINFORMULA} \in NP$ and since we assumed $P = NP$, we have that $\overline{MINFORMULA} \in P$. \\

Since $P = coP$ and $\overline{MINFORMULA} \in P$, we have that $\overline{MINFORMULA} \in coP$. \\

Since $\overline{MINFORMULA} \in coP$, we have that $MINFORMULA \in P$, as required.

\end{proof}

\newpage

\section*{Question 4}

\textbf{Required:} Show $BIPARTITE \in NL$. \\

We are given the hint that a graph is not bipartite if and only if it has a cycle with an odd number of vertices. \\

We also know that $NL = coNL$. Hence, it is sufficient to just show that $BIPARTITE \in coNL$. i.e. We must show that $\overline{BIPARTITE} \in NL$. \\

Consider the following nondeterministic log space TM $M$ that decides $\overline{BIPARTITE}$. \\

$M = $ "on input $\langle G \rangle$: \\

\setlength\parindent{15pt}

\indent 1. Keep a counter $c$ starting at $c = 0$. \\
 
\indent 2. Nondeterministically select some initial vertex $s$ from $G$. \\

\indent 3. Nondeterministically select some vertex $s^\prime$ connected to $s$. Increment the counter so that $c \leftarrow c + 1$. \\

\indent 4. Repeat the following so long as $c < |V|$ where $V$ is the set of vertices of $G$ and so long as $s^\prime \neq s$: \\

\indent \indent \indent i) Nondeterministically select some vertex $s^{\prime \prime}$ connected to $s^\prime$. Let $s^\prime \leftarrow s^{\prime \prime}$. \\
\indent \indent \indent ii) Increment the counter so that $c \leftarrow c + 1$. \\
\indent \indent \indent iii) If $s^\prime = s$, then terminate the loop and move on to Step 5. \\

\indent 5. If $s = s^\prime$ and $c$ is odd, then accept. Otherwise, reject." \\

\setlength\parindent{0pt}

And we implicitly assume that if $M$ is given a string $w$ that is not an encoding of some graph $G$, then $M$ rejects $w$. \\ 

Clearly $M$ decides $\overline{BIPARTITE}$ given the hint that a graph is not bipartite if and only if it has a cycle with an odd number of vertices. \\

To see that $M$ runs in log space, notice that the counter $c$ is just a binary number less than or equal to the number of vertices in $G$. Hence, $c$ can be stored in log space with respect to the input length. And the inital vertex $s$ can be stored in log space with respect to the input length. The space used to store $s^\prime$ is log space with respect to the input length, and this space is reused each time we update $s^\prime$. Hence, $M$ runs in log space with respect to the input length. \\

Therefore, $\overline{BIPARTITE} \in NL$. Hence, $BIPARTITE \in coNL$. Since $NL = coNL$, we conclude that $BIPARTITE \in NL$. 

\end{document} 




