\documentclass[12pt]{article}

\usepackage{amsmath}
\usepackage{amsthm}
\usepackage[margin=2.5cm]{geometry}
\usepackage{amsfonts}
\usepackage{amssymb}

\newcommand{\N}{\mathbb{N}}
\newcommand{\Z}{\mathbb{Z}}
\newcommand{\Q}{\mathbb{Q}}
\newcommand{\R}{\mathbb{R}}
\newcommand{\C}{\mathbb{C}}

\def\plus{\text{+}}

% Document starts here

\begin{document}

\pagenumbering{gobble}

\setlength\parindent{0pt}

\section*{Question 1}

\textbf{Required:} Show that if $P = NP$, then every language $A \in P$ except $A = \emptyset$ and $A = \Sigma^*$ is NP-complete.

\begin{proof}
Assume $P = NP$. Let $A \in P$ such that $A \neq \emptyset$ and $A \neq \Sigma^*$. \\

\textbf{Want to Show:} $A$ is NP-complete. \\

Since $A \in P$ and $P = NP$, we have that $A \in NP$. \\

Since $A \in P$, we know there exists some TM $M$ such that $M$ decides $A$ in polynomial time. \\

Since $A \neq \emptyset$ and $A \neq \Sigma^*$, we know there exists some $w_1 \in A$ and there exists some $w_2 \not \in A$. \\

Now let $B \in NP$ be arbitrary. Consider the following TM $F$. \\

$F = $ "on input $w$: \\

\setlength\parindent{15pt}

\indent 1. Run $M$ on $w$. \\
\indent 2. If $M$ accepts $w$, then output $w_1$ on the tape. \\
\indent 3. If $M$ rejects $w$, then output $w_2$ on the tape." \\

\setlength\parindent{0pt}

Clearly $F$ runs in polynomial time as Step 1 simulates $M$ which runs in polynomial time with respect to the length of the input, and Steps 2 and 3 trivially runs in polynomial time. \\

So $F$ induces a computable function $f: \Sigma^* \rightarrow \Sigma^*$ where $F$ runs in polynomial time. \\

And clearly $w \in B$ if and only if $f(w) \in A$. Hence, we have $B \leq_p A$. \\

Since $B \in NP$ was arbitrary, we have that every $B \in NP$ is such that $B \leq_p A$. So $A$ is $NP$-hard. \\

Since $A \in P$ and $P = NP$, we have that $A \in NP$. \\

Since $A$ is $NP$-hard and $A \in NP$, we have that $A$ is NP-complete, as required. 
\end{proof}

\newpage

\section*{Question 2}

\textbf{Required:} Show that $TAUT$ is $coNP$-complete (i.e. $TAUT \in coNP$ and there is a reduction $A \leq_p TAUT$ for all $A \in coNP$. 

\begin{proof}
First we will show that $TAUT \in coNP$. Hence, we must show that $\overline{TAUT} \in NP$. \\

Consider the following polynomial time verifier $V$ for $\overline{TAUT}$ which considers a falsifying truth assignment as a certificate. \\

$V = $ "on input $\langle \phi, c \rangle$: \\

\setlength\parindent{15pt}

\indent 1. Test that $\phi$ is indeed a well-formed formula and that $c$ is an encoding of an assignment of truth values to all the variables in $\phi$. \\
\indent 2. Test that $\phi$ is false under the truth assignment that $c$ encodes. \\
\indent 3. If both tests pass, then accept. Otherwise, reject." \\

\setlength\parindent{0pt}
If $\langle \phi \rangle \in \overline{TAUT}$, then there exists a truth assignment that makes $\phi$ false. Hence, a certificate $c$ exists which encodes this assignment and $V$ accepts $\langle \phi, c \rangle$. If $\langle \phi \rangle \not \in \overline{TAUT}$, then $\phi$ is a tautology and there does not exist a truth assignment that makes $\phi$ false, and so no such certificate exists. And clearly $V$ is a decider as it always halts. Hence, $V$ is a verifier. \\

Clearly $V$ is a poly-time verifier as the number of variables of $\phi$ is less than $|\langle \phi \rangle|$. And a valid certificate is just an assignment of truth values to these variables that make $\phi$ false. And checking that this assignment falsifies $\phi$ also takes polynomial time with respect to $|\langle \phi \rangle|$. \\

Since we have a poly-time verifier for $\overline{TAUT}$, we have that $\overline{TAUT} \in NP$. Hence, we have that $TAUT \in coNP$. \\

Now consider the following reduction from $SAT$ to $\overline{TAUT}$ given by the following TM $F$. \\

$F = $ "on input $\langle \phi \rangle$: \\

\setlength\parindent{15pt}

\indent 1. Output $\langle \neg \phi \rangle$" \\

\setlength\parindent{0pt}
And if $F$ is given an input that is not an encoding of a well-formed formula $\phi$, then output $\langle x_1 \lor \overline{x_1} \rangle$ where $x_1 \lor \overline{x_1}$ is a tautology. \\

Hence, $F$ induces the the following computable function $f: \Sigma^* \rightarrow \Sigma^*$. \\
\[
  f(x) =
  \begin{cases}
        \langle \neg \phi \rangle & \text{if $x = \langle \phi \rangle$} \\
        \langle x_1 \lor \overline{x_1} \rangle & \text{otherwise} 
  \end{cases}
\]
Note that if $\phi$ is any satisfiable CNF formula, we have that $\neg \phi$ cannot be a tautology since any assignment that makes $\phi$ true would make $\neg \phi$ false. \\

Hence, we get $x \in SAT$ if and only if $f(x) \in \overline{TAUT}$. And trivially $F$ runs in polynomial time as we are just adding a negation sign $\neg$ in front of $\phi$. Hence, $f$ is a polynomial reduction. \\

Hence, $SAT \leq_p \overline{TAUT}$. \\

Now let $A \in coNP$. Hence, $\overline{A} \in NP$. We know that $SAT$ is NP-complete from lecture and the textbook. Hence, $\overline{A} \leq_p SAT$. Since $\overline{A} \leq_p SAT$ and $SAT \leq_p \overline{TAUT}$, by transitivity of $\leq_p$ we get that $\overline{A} \leq_p \overline{TAUT}$. \\

From $\overline{A} \leq_p \overline{TAUT}$, we want to show $A \leq_p TAUT$. Since $\overline{A} \leq_p \overline{TAUT}$, there exists a polynomial reduction $g$ such that $w \in \overline{A} \Leftrightarrow g(w) \in \overline{TAUT}$. Equivalently, $w \not \in A \Leftrightarrow g(w) \not \in TAUT$. Equivalently, via contrapositive we get $w \in A \Leftrightarrow g(w) \in TAUT$. Hence, $A \leq_p TAUT$. \\

Since $A \in coNP$ was arbitrary, we have that every $A \in coNP$ is such that $A \leq_p TAUT$. Since we've shown earlier that $TAUT \in coNP$, we have that $TAUT$ is $coNP$-complete, as required. \\

\end{proof}

\newpage

\section*{Question 3}

Let $CNF_k$ be the set of all satisfiable $CNF$ formulas where each variable appears at most $k$ times. \\

\textbf{Required:} Show that $CNF_2 \in P$. 

\subsection*{Defining a $TM$ $M$ that decides $CNF_2$}

Consider the following TM $M$ that decides $CNF_2$ in polynomial time. \\

$M = $ "on input $\langle \phi \rangle$: \\

\setlength\parindent{15pt}

\indent 1. Test that $\phi$ is $CNF$ and each variable in $\phi$ appears at most 2 times. If this test fails, reject immediately. \\

\indent 2. Let $x_1,x_2,...,x_k$ be the variables that occur in $\phi$. For each $i \in \{1,...,k\}$, repeat the following sub-steps i), ii), iii), and iv). \\

i) If $x_i$ occurs exactly once in $\phi$ in some clause $c_i$, then modify $\phi$ so that the clause $c_i$ containing $x_i$ is removed. Note that $x_i$ could appear as $x_i$ or $\overline{x_i}$ in such a clause $c_i$. \\

ii) If $x_i$ occurs exactly twice in $\phi$, where both occurrences are $x_i$ or both occurrences are $\overline{x_i}$, then modify $\phi$ so that the clause(s) containing these occurrences are removed. \\

iii) If $\phi$ contains an occurence of $x_i$ and an occurrence of $\overline{x_i}$ in the same clause $c_i$, then remove this clause. \\

iv) If $\phi$ contains an occurence of $x_i$ and an occurrence of $\overline{x_i}$ in distinct clauses $c_{i_1}$ and $c_{i_2}$, then consider the following modification to $\phi$. If $c_{i_1} = x_i$ and $c_{i_2} = \overline{x_i}$, then do not modify $\phi$. If $c_{i_1} \neq x_i$ or $c_{i_2} \neq \overline{x_i}$, then modify $\phi$ by replacing the causes $c_{i_1}$ and $c_{i_2}$ with the clause $c^\prime$, where $c^\prime$ is the clause containing all the literals of $c_{i_1}$ and $c_{i_2}$ except for $x_i$ and $\overline{x_i}$. \\

\indent 3. If $\phi = \epsilon$, then accept. Otherwise, reject." 
\setlength\parindent{0pt}

\subsection*{Explanation of $M$}

It is clear that $M$ always halts, and hence is a decider. To see that $M$ decides $CNF_2$ in polynomial time, first we will clarify what $M$ in fact accomplishes. \\

Step 1 simply checks that $\phi$ is indeed $CNF$ where each variable in $\phi$ appears at most 2 times. \\

Step 2 then reduces $\phi$ by removing and modifying clauses of $\phi$. Please see the next page. \\

Variables $x_i$ that occur exactly once in $\phi$ can have 1 assigned to $x_i$ (or 0 to $x_i$ if $\overline{x_i}$ appears) to make their respective clauses satisfied. Hence, we can reduce $\phi$ by removing these clauses which is what Step 2 i) does. The satisfiability of the original $\phi$ is unchanged. \\

Variables $x_i$ that occur twice (either both $x_i$ or both $\overline{x_i}$) can have (1 or 0 respectively) assigned to $x_i$ to make their respective clause(s) satisfied. Hence, we can reduce $\phi$ by removing these clause(s) which is what Step 2 ii) does. The satisfiability of the original $\phi$ is unchanged. \\

Variables $x_i$ that occur twice (one occurrence of $x_i$ and one occurence of $\overline{x_i}$) within the same clause can have $x_i$ assigned 1 to make this respective clause satisfied. Hence, we can reduce $\phi$ by removing this clause which is what Step 2 iii) does. The satisfiability of the original $\phi$ is unchanged. \\

Variables $x_i$ that occur twice (one occurrence of $x_i$ and one occurence of $\overline{x_i}$) in distinct clauses $c_{i_1}$ and $c_{i_2}$ cause $\phi$ to be modified as follows. If the occurrences of $x_i$ and $\overline{x_i}$ are the only literals in their respective clauses, then $\phi$ is unchanged. Otherwise, we replace the causes $c_{i_1}$ and $c_{i_2}$ with the clause $c^\prime$, where $c^\prime$ is the clause containing all the literals of $c_{i_1}$ and $c_{i_2}$ except for $x_i$ and $\overline{x_i}$. Note that this does not change the satisfiability of $\phi$ since it is clear that $c^\prime$ is satisfiable if and only if  $c_{i_1} \land c_{i_2}$ is satisfiable. This is the case since any assignment that satisfies $c^\prime$ must satisfy at least one literal that does not contain the variable $x_i$. Hence, at least one of $c_{i_1}$ or $c_{i_2}$ would be satisfied. WLOG, suppose $c_{i_1}$ is satisfied. Since $c_{i_2}$ contains $x_i$ or $\overline{x_i}$, assign 1 (or 0) respectively to $x_i$ to satisfy $c_{i_2}$. Hence, we would satisfy $c_{i_1} \land c_{i_2}$. Conversely, any assignment that satisfies $c_{i_1} \land c_{i_2}$ must satisfy some literal that is not $x_i$ or $\overline{x_i}$. This literal would also be part of $c^\prime$, and hence $c^\prime$ would be satisfied. Hence, the satisfiability of the now reduced $\phi$ is also unchanged. \\


When we reach Step 3, our modified $\phi$ will be in one of two cases. In the first case, $\phi$ will be an empty formula (no connectives or variables). In the second case, $\phi$ will contain clauses with 1 literal each, and $\phi$ will contain pairs of contradictory clauses like $c^\prime = x_i$ and $c^{\prime \prime} = \overline{x_i}$ which is unsatisfiable since $x_i \land \overline{x_i}$ is unsatisfiable. \\

Hence, if $\langle \phi \rangle \in CNF_2$, then $\phi$ will be a satisfiable $CNF$ formula where each variable appears at most twice, and $M$ will reduce $\phi$ to the empty string and be accepted. \\

And, if $\langle \phi \rangle \not \in CNF_2$, then $\phi$ is either not a $CNF$ formula where each variable appears at most twice and will be rejected at Step 1, or $\phi$ is not satisfiable and on Step 3 will be reduced to a formula containing contradictory clauses $x_i$ and $\overline{x_i}$ for some variable $x_i$, and will be rejected. \\

Each step in $M$ clearly takes polynomial time with respect to $|\langle \phi \rangle|$ since each step simply scans through the string $\langle \phi \rangle$ and possibly does some deletion of the string $\langle \phi \rangle$ as we modify and delete the clauses of $\phi$. Hence, $M$ runs polynomial in $|\langle \phi \rangle|$. \\

Therefore, $CNF_2 \in P$, as required. 

\newpage

\section*{Question 4}

\textbf{Required:} Show that $CNF_3$ is $NP$-complete. 

\subsection*{Show $CNF_3 \in NP$}

Consider the following verifier $V$ for $CNF_3$ which uses a satisfying assignment as a certificate. \\

$V = $ "on input $\langle \phi, c \rangle$: \\

\setlength\parindent{15pt}

\indent 1. Test that $\phi$ is a CNF formula where each variable appears at most 3 times. \\
\indent 2. Test that $c$ is an encoding of an assignment of truth values to all the variables in $\phi$. \\
\indent 3. Test that $\phi$ is true under the truth assignment that $c$ encodes. \\
\indent 4. If all tests pass, then accept. Otherwise, reject." \\
\setlength\parindent{0pt}

Clearly $V$ is a decider as it always halts. If $\langle \phi \rangle \in CNF_3$, then $\phi$ is a satisfiable CNF formula where each variable appears at most 3 times. Hence, a satisfying assignment exists which serves a certificate $c$, and $V$ will accept $\langle \phi, c \rangle$. If $\langle \phi \rangle \not \in CNF_3$, then either $\phi$ is not CNF, or it is not the case that every variable appears at most 3 times, or $\phi$ is not satisfiable. Hence, no such certificate exists. \\

Clearly $V$ is a poly-time verifier as the number of variables of $\phi$ is less than $|\langle \phi \rangle|$. And a valid certificate is just an assignment of truth values to these variables that make $\phi$ true. And checking that this assignment satisfies $\phi$ also takes polynomial time with respect to $|\langle \phi \rangle|$. \\

Hence, $CNF_3 \in NP$. 

\subsection*{Show $CNF_3$ is NP-Hard} \\

We know from lecture and the textbook that $SAT_3$ is NP-Complete. We will show that $SAT_3 \leq_p CNF_3$. \\

Let $\alpha$ be a CNF formula with three literals per clause. \\

Let $Vars_{\alpha} = \{x_1,x_2,x_3,...,x_k\}$ be the set of variables that appear in $\alpha$. Note that a variable $x_i \in Vars_{\alpha}$ could appear more than 3 times in $\alpha$. \\

Hence, we need to come up with a new formula $\alpha^\prime$, where each variable in $\alpha^\prime$ appears at most 3 times such that $\alpha$ is satisfiable if and only if $\alpha^\prime$ is satisfiable. \\

First let $\beta$ be the same formula as $\alpha$ except that all the variables are renamed so that each variable appears exactly once. \\

We need to add additional conjunctive clauses to $\beta$ so that variables that were intially the same variable in $\alpha$ are always assigned the same value in any satisfying assignment. \\

i.e. Let $x_i \in Vars_{\alpha}$. Assume that there are $j$ occurrences of $x_i$ in $\alpha$. Then each occurrence of $x_i$ is renamed in $\beta$. Let the renamed occurrences of $x_i$ in $\beta$ be $y_{i_{1}},y_{i_2},...,y_{i_j}$. \\

We want to enforce the fact that all the renamed variables $y_{i_{1}},y_{i_2},...,y_{i_j}$ are all given the same value (true or false) in any satisfying assignment. \\

To do this, consider the following formula. 
\begin{align*}
(y_{i_{1}} \Rightarrow y_{i_{2}}) \land (y_{i_{2}} \Rightarrow y_{i_{3}}) \land \cdot \cdot \cdot \land (y_{i_{j-1}} \Rightarrow y_{i_{j}}) \land (y_{i_{j}} \Rightarrow y_{i_{1}})
\end{align*}
The above formula is clearly satisfied only when $y_{i_{1}},y_{i_2},...,y_{i_j}$ are all assigned True, or are all assigned False since the formula is a closed chain of implications. But we need this formula to be in $CNF$. But we can easily translate this formula by translating each implication to the disjunction connective using the logical equivalence $\phi \Rightarrow \psi \equiv \neg \phi \lor \psi$. Hence, we get the following equivalent formula. 
\begin{align*}
(\overline{y_{i_{1}}} \lor y_{i_{2}}) \land (\overline{y_{i_{2}}} \lor y_{i_{3}}) \land \cdot \cdot \cdot \land (\overline{y_{i_{j-1}}} \lor y_{i_{j}}) \land (\overline{y_{i_{j}}} \lor y_{i_{1}})
\end{align*}
So let $\gamma_{x_i} := (\overline{y_{i_{1}}} \lor y_{i_{2}}) \land (\overline{y_{i_{2}}} \lor y_{i_{3}}) \land \cdot \cdot \cdot \land (\overline{y_{i_{j-1}}} \lor y_{i_{j}}) \land (\overline{y_{i_{j}}} \lor y_{i_{1}})$. \\

Note that $\gamma_{x_i}$ is CNF and each variable in $\gamma_{x_i}$ appears exactly twice. \\

Since we need to enforce the above for each variable $x_i \in Vars_{\alpha}$, consider the conjunction $\gamma := \bigwedge\limits_{x_i \in Vars_{\alpha}}\gamma_{x_i}$ \\

So $\beta \land \gamma$ is our desired formula which is in CNF. Since each variable in $\beta$ appears exactly once, and each variable in $\gamma$ appears exactly twice, we have that each variable in $\beta \land \gamma$ appears exactly 3 times. \\

Let $\alpha^\prime := \beta \land \gamma$. \\

\textbf{Want to Show:} $\langle \alpha \rangle \in SAT_3$ if and only if $\langle \alpha^\prime \rangle \in CNF_3$ \\

$(\Rightarrow):$ If $\langle \alpha \rangle \in SAT_3$, then there exists a truth assignment $v$ that makes $\alpha$ true. Consider a new assignment $v^\prime$ that assigns the same truth value that $v$ assigns to each $x_i \in Vars_{\alpha}$ to each of $x_i$'s renamed variables in $\alpha^\prime$. Hence, $v^\prime$ satisfies $\beta$. And since each renamed variable of each occurence of $x_i$ is given the same truth value for each variable $x_i$ in $\alpha$, we have that $\gamma$ is also satisfied. Hence, $\alpha^\prime := \beta \land \gamma$ is satisfied. Hence, $\langle \alpha^\prime \rangle \in CNF_3$. \\

$(\Leftarrow):$ If $\langle \alpha^\prime \rangle \in CNF_3$, then there exists a truth assignment $v^\prime$ that makes $\alpha^\prime$ true. Given how $\gamma$ was constructed, we know that the group of variables $y_{i_{1}},y_{i_2},...,y_{i_j}$ that were renamings of the same variable $x_i$ from $\alpha$ must have the same truth value (all True or all False). Hence, let $v$ be an assignment that assigns each $x_i \in Vars_{\alpha}$ the same truth value as $v^\prime$ assigns to each of $y_{i_{1}},y_{i_2},...,y_{i_j}$. Hence, $v$ makes $\alpha$ true so that $\alpha$ is satisfiable. Hence, $\langle \alpha \rangle \in SAT_3$. \\

Please see the next page. 

\newpage

Now, consider the following $TM$ $F$ that induces a computable function $f$. \\

$F = $ "on input $\langle \alpha \rangle$: \\

\setlength\parindent{15pt}

\indent 1. Check that $\alpha$ is CNF with at most 3 literals per clause. If not, output $\langle x_1 \land \overline{x_1} \rangle$. \\

\indent 2. Otherwise, construct the formula $\alpha^\prime := \beta \land \gamma$ as outlined in the above discussion. \\

\indent 3. Output $\langle \alpha^\prime \rangle$." \\
\setlength\parindent{0pt}

Notice, if $F$ is given a string that is not a valid input, we simply output $\langle x_1 \land \overline{x_1} \rangle$, where $x_1 \land \overline{x_1}$ is a CNF formula where each variable appears at most 3 times and is not satisfiable. \\

Hence, $F$ induces the following function $f: \Sigma^* \rightarrow \Sigma^*$. \\
\[
  f(w) =
  \begin{cases}
        \langle \alpha^\prime\rangle & \text{if $w = \langle \alpha \rangle$, where $\alpha$ is CNF with 3 literals per clause} \\
        \langle x_1 \land \overline{x_1} \rangle & \text{otherwise} \\
  \end{cases}
\] 
\\
Recall, we showed earlier that $\langle \alpha \rangle \in SAT_3$ if and only if $\langle \alpha^\prime \rangle \in CNF_3$. \\

Hence, $w \in SAT_3$ if and only if $f(w) \in CNF_3$. \\

And clearly $F$ is polynomial as each step only takes polynomial steps with respect to $|\langle \alpha \rangle|$. i.e. We only take polynomial steps with respect to $|\langle \alpha \rangle|$ in constructing $\alpha^\prime$. \\
 
Hence, $f$ is a polynomial reduction. Hence, $SAT_3 \leq_p CNF_3$. \\

From lecture and the textbook we know that $SAT_3$ is NP-complete. \\

Let $A \in NP$. Hence, $A \leq_p SAT_3$. Since $A \leq_p SAT_3$ and $SAT_3 \leq_p CNF_3$, by transitivity of $\leq_p$ we get that $A \leq_p CNF_3$. Since $A \in NP$ was arbitrary, we get that $CNF_3$ is $NP$-hard. \\

Since $CNF_3$ is $NP$-hard and since we showed earlier that $CNF_3 \in NP$, we conclude that $CNF_3$ is $NP$-complete, as required. 

\newpage

\section*{Question 5}

\textbf{Required:} Show that $TF$ is $NP$-complete. \\

First we will show that $TF \in NP$. \\

Consider the following verifier $V_0$ which uses a triangle-free subset as a certificate. \\

$V_0 = $ "on input $\langle \langle G, k \rangle, c \rangle$: \\

\setlength\parindent{15pt}

\indent 1. Test that $G = (V, E)$ for some set of vertices $V$ and set of edges $E$. \\
\indent 2. Test that $c = \langle S \rangle$ is an encoding of some subset $S \subseteq V$ such that $|S| \geq k$. \\
\indent 3. For every 3 element subset $\{x,y,z\} \subseteq S$, test that at least one of the possible edges between any two distinct vertices in $\{x,y,z\}$ is not an element of $E$. \\
\indent 4. If all tests pass, then accept. Otherwise, reject." \\
\setlength\parindent{0pt}

Clearly $V_0$ is a decider since it always halts. And if $\langle G, k \rangle \in TF$, then $G$ has a triangle-free subset $S$ of size at least $k$. Hence, all tests will clearly pass and $V_0$ will accept $\langle \langle G, k \rangle, c \rangle$ where $c = \langle S \rangle$. And if $\langle G, k \rangle \not \in TF$, then $G$ does not have a subset of size at least $k$ that is triangle-free. Hence, no such certificate exists. \\

To show that $V_0$ runs in polynomial time, notice that Steps 1 and 2 clearly runs in polynomial time with respect to the input length. And Step 3 considers any $3$ element subset of $S$ where $S \subseteq V$ and $|S| \geq k$. Hence, the number of subsets we're considering is $\binom{|S|}{3} \leq \binom{|V|}{3} = \frac{|V|!}{3!(|V|-3)!} = \frac{|V|(|V|-1)(|V|-2)}{3!} \in O(|V|^3)$. Since this is polynomial with respect to $|V|$, we have that Step 3 is clearly polynomial with respect to the length of the input. Hence, $V_0$ is polynomial. Since we have a polynomial verifier $V_0$ for $TF$, we have that $TF \in NP$. \\


\textbf{Notation:} Since we are working with undirected graphs, we will represent an edge between vertices $x$ and $y$ by a set $\{x,y\}$. This is consistent with Professor Saraf's notation from the lecture slides. \\

As the hint suggests, we will reduce from $INDSET$. \\

Let $G = (V, E)$ be a graph for some set of vertices $V$ and set of edges $E$. \\

Let $V_0$ be a set of vertices such that $V \cap V_0 = \emptyset$ and $|V_0| = |V| + 1$. \\

Let $V^\prime = V \cup V_0$. Note, $|V^\prime| = |V| + |V_0| = |V| + |V| + 1 = 2|V| + 1$ \\

Let $E^\prime = E \cup \{\{x,y\} : x \in V \land y \in V_0\}$. \\

Consider the graph $G^\prime = (V^\prime, E^\prime)$. \\

Let $k \in \N$. \\

\textbf{Want to Show:} Given $G$ and $G^\prime$ as defined, we have that  $\langle G, k \rangle \in INDSET$ if and only if $\langle G^\prime, k + |V| + 1 \rangle \in TF$ \\

($\Rightarrow$): Assume $\langle G, k \rangle \in INDSET$. Hence, $G$ has an independent set $S \subseteq V$ such that $|S| \geq k$. \\

Now consider the set $S \cup V_0$. Since $S \subseteq V$ and $V \cap V_0 = \emptyset$, we have that $S \cap V_0 = \emptyset$. Hence, $|S \cup V_0| = |S| + |V_0| \geq k + |V| + 1$. \\

And we have $S \cup V_0 \subseteq V^\prime$. We just need to show that $S \cup V_0$ is triangle-free. Let $\{x,y,z\} \subseteq S \cup V_0$. \\

\textbf{Case 1:} Assume $x,y,z \in V_0$. Looking at the definition of $E^\prime$, since $x,y \in V_0$ and $V_0 \cap V = \emptyset$, we have that $\{x,y\} \not \in E^\prime$. Hence, there is no triangle involving $x,y,z$ in $G^\prime$. \\

\textbf{Case 2:} Assume exactly two of $x,y,z$ are contained in $V_0$. WLOG, assume $x,y \in V_0$. Looking at the definition of $E^\prime$, since $x,y \in V_0$ and $V \cap V_0 = \emptyset$, we have that $\{x,y\} \not \in E^\prime$. Hence, there is no triangle involving $x,y,z$ in $G^\prime$. \\

\textbf{Case 3:} Assume exactly one of $x,y,z$ is contained in $V_0$. WLOG, assume $x \in V_0$ so that $y,z \in S$. Since $y,z \in S$, and $S$ is an independent set, we know that $\{y,z\} \not \in E$. We know that $\{y,z\} \not \in \{\{x,y\} : x \in V \land y \in V_0\}$ since $y,z \in S \subseteq V$ and $V \cap V_0 = \emptyset$. Hence, $\{y,z\} \not \in E^\prime$. Hence, there is no triangle involving $x,y,z$ in $G^\prime$. \\

In all three cases we have no triangle involving $x,y,z$ in $G^\prime$. Therefore, $S \cup V_0$ is a triangle-free set such that $|S \cup V_0| \geq k + |V| + 1$. Hence, $\langle G^\prime, k + |V| + 1 \rangle \in TF$. \\

($\Leftarrow$): Assume $\langle G^\prime, k + |V| + 1 \rangle \in TF$. Hence, $G^\prime$ has a triangle-free set $S^\prime \subseteq V^\prime$ such that $|S^\prime| \geq k + |V| + 1$. \\

Let $S = S^\prime \setminus V_0$. Notice,
\begin{align*}
|S| &= |S^\prime \setminus V_0| \\
&\geq |S^\prime| - |V_0| &&\text{By Properties of Sets} \\
&\geq (k + |V| + 1|) - |V_0| &&\text{Since $|S^\prime| \geq k + |V| + 1$} \\
&= (k + |V| + 1) - (|V| + 1) &&\text{Since $|V_0| = |V| + 1$} \\
&= k
\end{align*}
Hence, $|S| \geq k$. And $S \subseteq V$ by definition. We want to show that $S \subseteq V$ is an independent set in $G$. \\

Let $x,y \in S$ be distinct vertices. We want to show that $\{x,y\} \not \in E$. \\

Assume for contradiction that $\{x,y\} \in E$. Since $E \subseteq E^\prime$, we have that $\{x,y\} \in E^\prime$. \\

Note that $|S^\prime| \geq k + |V| + 1$. And $S^\prime \subseteq V^\prime = V \cup V_0$, where $|V_0| = |V| + 1$ and $V \cap V_0 = \emptyset$. Hence, there must exist a $z \in S^\prime$ such that $z \in V_0$. \\

Since $x,y \in S \subseteq V$ and $z \in V_0$, looking at $E^\prime = E \cup \{\{x,y\} : x \in V \land y \in V_0\}$, we get that $\{x,z\} \in E^\prime$ and $\{y,z\} \in E^\prime$. \\

So we have $\{x,y\}, \{x,z\}, \{y,z\} \in E^\prime$. But we know that $x,y \in S \subseteq S^\prime$. And $z \in S^\prime$. So $x,y,z \in S^\prime$ where $x,y,z$ form a triangle, contradicting the fact that $S^\prime$ is triangle-free. \\

Therefore, our initial assumption was wrong and $\{x,y\} \not \in E$. Hence, we've shown that $S$ is an independent set. \\

Now consider the following TM $F$ which induces a computable function $f$. \\

$F = $ "on input $\langle G, k \rangle$: \\

\setlength\parindent{15pt}

\indent 1. Check that $G = (V, E)$ for some set of vertices $V$ and set of edges $E$. If not, output $\langle H, 3 \rangle$, where $H$ is a graph with exactly 3 vertices $x,y,z$ with edges connected pairwise as a triangle. \\

\indent 2. Otherwise, construct the graph $G^\prime$ as outlined in the above discussion. \\

\indent 3. Output $\langle G^\prime, k + |V| + 1 \rangle$." \\
\setlength\parindent{0pt}

Note that if $F$ is given an input that is not a valid graph, we output $\langle H, 3 \rangle$, where $H$ is a graph with exactly 3 vertices with edges connected as a triangle. Hence, $\langle H, 3 \rangle \not \in TF$. \\

Hence, $F$ induces the following function $f: \Sigma^* \rightarrow \Sigma^*$. \\
\[
  f(w) =
  \begin{cases}
        $\langle G^\prime, k + |V| + 1 \rangle$ & \text{if $G = (V, E)$ for some set of vertices $V$ and set of edges $E$} \\
        $\langle H, 3 \rangle$ & \text{otherwise} \\
  \end{cases}
\]
And we've shown earlier that $\langle G, k \rangle \in INDSET$ if and only if $\langle G^\prime, k + |V| + 1 \rangle \in TF$. \\

Hence, $w \in INDSET$ if and only if $f(w) \in TF$. \\

We know $F$ is polynomial-time since constructing $G^\prime$ is trivially proportional polynomially to the length $|\langle G \rangle|$. And constructing $H$ is always constant. \\

So $f$ is a polynomial reduction. Hence, $INDSET \leq_p TF$. \\

We know from lecture that $INDSET$ is $NP$-complete. So let $A \in NP$. Hence, $A \leq_p INDSET$. Since $A \leq_p INDSET$ and $INDSET \leq_p TF$, we have that $A \leq_p TF$. Since $A$ was arbitrary, we have that $TF$ is $NP$-hard. \\

Since $TF$ is $NP$-hard, and since we've shown earlier that $TF \in NP$, we have that $TF$ is $NP$-complete, as required.  










\end{document} 




