\documentclass[12pt]{article}

\usepackage{amsmath}
\usepackage{amsthm}
\usepackage[margin=2.5cm]{geometry}
\usepackage{amsfonts}
\usepackage{amssymb}

\newcommand{\N}{\mathbb{N}}
\newcommand{\Z}{\mathbb{Z}}
\newcommand{\Q}{\mathbb{Q}}
\newcommand{\R}{\mathbb{R}}
\newcommand{\C}{\mathbb{C}}

\def\plus{\text{+}}

% Document starts here

\begin{document}

\pagenumbering{gobble}

\setlength\parindent{0pt}

\section*{Question 1}

\subsection*{Part a)}

\textbf{Required:} Show that a language $B$ is Turing-recognizable if and only if $B \leq_m A_{TM}$. 

\begin{proof}

$(\Rightarrow):$ Assume $B$ is a Turing-recognizable language. Hence, there exists a TM $M$ that recognizes $B$. \\

First we will define a mapping reduction $f$ from $B$ to $A_{TM}$. \\

The mapping reduction $f$ will be defined based on the TM $F$ given below. \\

$F = $ "On input $w$: \\

\setlength\parindent{15pt}

\indent 1. Output $\langle M, w \rangle$." \\

\setlength\parindent{0pt}

Hence, $f(w) = \langle M, w \rangle$, where $M$ is the TM that recognizes $B$. \\

Clearly, $w \in B$ if and only if $\langle M, w \rangle \in A_{TM}$. \\

Hence, $w \in B$ if and only if $f(w) \in A_{TM}$. Therefore, $B \leq_m A_{TM}$. \\

$(\Leftarrow):$ Assume $B \leq_m A_{TM}$. Hence, there exists some computable function $f: \Sigma^* \rightarrow \Sigma^*$ such that for every $w \in \Sigma^*$, we have that $w \in B$ if and only if $f(w) \in A_{TM}$. \\

We know from lecture and the textbook that $A_{TM}$ is recognizable. Let $M$ be the TM that recognizes $A_{TM}$. \\

Consider the following $TM$ $M^\prime$ that recognizes $B$. \\

$M^\prime = $ "On input $w$: \\

\setlength\parindent{15pt}

\indent 1. Compute $f(w)$. \\

\indent 2. Run $M$ on input $f(w)$. If $M$ accepts $f(w)$, then accept. If $M$ rejects $f(w)$, then reject." \\

\setlength\parindent{0pt}
To show that $M^\prime$ recognizes $B$, consider the following. Since $f$ is a mapping reduction from $B$ to $A_{TM}$, we know $w \in B$ if and only if $f(w) \in A_{TM}$. \\

If $w \in B$, then $f(w) \in A_{TM}$. So $M$ will accept $f(w)$, and hence $M^\prime$ will accept $w$. \\

If $w \not \in B$, then $f(w) \not \in A_{TM}$. So $M$ will not accept $f(w)$, and hence $M^\prime$ will not accept $w$. Therefore, $M^\prime$ recognizes $B$. \\

Since $M^\prime$ recognizes $B$, we have shown that $B$ is recognizable.
\end{proof}

\subsection*{Part b)}

\textbf{Required:} Show that a language $A$ is decidable if and only if $A \leq_m 0^*1^*$ where $0^*1^*$ is the language containing strings of an arbitrary number of 0s, followed by an arbitrary number of 1s

\begin{proof}
($\Rightarrow$): Assume $A$ is decidable. Hence, there is a TM $M$ that decides $A$. Consider the following mapping reduction $f$ from $A$ to $0^*1^*$ that is given by the following TM $F$. \\

$F = $ "On input $w$: \\

\setlength\parindent{15pt}

\indent 1. Run $M$ on $w$.  \\
\indent 2. If $M$ accepts $w$, output $01$. If $M$ rejects $w$, output $10$." \\

\setlength\parindent{0pt}
Hence, $f: \Sigma^* \rightarrow \Sigma^*$ is the following function. 

\[
  f(w) =
  \begin{cases}
        01 & \text{if $M$ accepts $w$} \\
        10 & \text{if $M$ rejects $w$} 
  \end{cases}
\]

Notice, $01 \in 0^*1^*$ and $10 \not \in 0^*1^*$. Hence, trivially we get $w \in B$ if and only if $f(w) \in 0^*1^*$. So $f$ is a mapping reduction. Hence, $A \leq_m 0^*1^*$. \\

($\Leftarrow$): Assume $A \leq_m 0^*1^*$. First we will show that $0^*1^*$ is decidable. Consider the following TM $M$ that decides $0^*1^*$. \\

$M = $ "On input $w$: \\

\setlength\parindent{15pt}

\indent 1. Scan $w$. If a $1$ appears before a $0$, then reject. Otherwise, accept." \\

\setlength\parindent{0pt}

To show that $M$ decides $0^*1^*$, first consider $w \in 0^*1^*$. Hence, $w$ does not contain a 1 that appears before a 0. Hence, $M$ will accept. If $w \not \in 0^*1^*$, then $w$ must contain a 1 that appears before a 0. Hence, $M$ will reject. Hence, $M$ is indeed a decider for $0^*1^*$. \\

Since $A \leq_m 0^*1^*$, we know there exists some computable function $f: \Sigma^* \rightarrow \Sigma^*$ such that for every $w \in \Sigma^*$, we have that $w \in A$ if and only if $f(w) \in 0^*1^*$. \\

Consider the following $TM$ $M^\prime$ that decides $A$. \\

$M^\prime = $ "On input $w$: \\

\setlength\parindent{15pt}

\indent 1. Compute $f(w)$. \\

\indent 2. Run $M$ on input $f(w)$. If $M$ accepts $f(w)$, then accept. If $M$ rejects $f(w)$, then reject." \\

\setlength\parindent{0pt}

To see that $M^\prime$ decides $A$, consider the following. Since $f$ is a mapping reduction from $A$ to $0^*1^*$, we know that $w \in A$ if and only if $f(w) \in 0^*1^*$. So if $w \in A$, we have that $f(w) \in 0^*1^*$, and so $M$ accepts $f(w)$. Hence, $M^\prime$ will accept $w$. If $w \not \in A$, then $f(w) \not \in 0^*1^*$. Hence, $M$ rejects $f(w)$ since $M$ decides $0^*1^*$. Hence, $M^\prime$ will also reject $w$. This shows that $M^\prime$ is a decider for $A$. \\

Therefore, $A$ is decidable. 
\end{proof}

\newpage

\section*{Question 2}

\subsection*{Part a)}

\textbf{Required:} Let $A = \{\langle M \rangle : \text{$M$ is a TM and } |L(M)| = 5\}$. Is $A$ Turing recognizable? Is $A$ Turing co-recognizable. \\

We will show that $A$ is neither Turing-recognizable nor Turing co-recognizable. \\

First we will define a mapping reduction $f$ from $A_{TM}$ to $A$. \\

The mapping reduction $f$ will be defined based on the TM $F$ given below. \\

$F = $ "On input $\langle M, w \rangle$: \\

\setlength\parindent{15pt}

\indent 1. Construct the following TM $M_w$. Let $w_1,w_2,w_3,w_4$ be the first four distinct strings of $\Sigma^*$ that appear in standard string order that are also distinct from $w$. \\
\indent \indent $M_w = $ 'On input $s$: \\
\indent \indent \indent 1. If $s = w_i$ for $i \in \{1,2,3,4\}$, then accept. \\
\indent \indent \indent 2. If $s = w$, then run $M$ on $w$ and output what $M$ outputs.' \\

\indent 2. Output $\langle M_w \rangle$". \\

\setlength\parindent{0pt}

And if $F$ is given a string that is not an encoding of a TM $M$ and a string $w$, then output $\langle M_{reject} \rangle$ where $M_{reject}$ rejects every string so that $|L(M_{reject})| = 0$. \\

Hence, 
\[
  f(x) =
  \begin{cases}
        \langle M_w \rangle & \text{if $x = \langle M, w \rangle$ for some TM $M$ and string $w$} \\
        \langle M_{reject} \rangle & \text{otherwise} 
  \end{cases}
\]

We have that $\langle M, w \rangle \in A_{TM}$ if and only if $f(\langle M, w \rangle) \in A$. \\

So we have a mapping reduction $f$ from $A_{TM}$ to $A$. Hence, $A_{TM} \leq_m A$. \\

Since $A_{TM} \leq_m A$, we also know that $\overline{A_{TM}} \leq_m \overline{A}$. Since we know that $\overline_{A_{TM}}$ is not recognizable from the textbook, we have that $\overline{A}$ is not recognizable. \\

Since $\overline{A}$ is not recognizable, we have that $A$ is not co-recognizable. \\

Next we will define a mapping reduction $g$ from $A_{TM}$ to $\overline{A}$. \\

The mapping reduction $g$ will be defined based on the TM $G$ given below. \\

$G = $ "On input $\langle M, w \rangle$: \\

\setlength\parindent{15pt}

\indent 1. Construct the following TM $M_w$. Let $w_1,w_2,w_3,w_4,w_5$ be the first five distinct strings of $\Sigma^*$ that appear in standard string order that are also distinct from $w$. \\
\indent \indent $M_w = $ 'On input $s$: \\
\indent \indent \indent 1. If $s = w_i$ for $i \in \{1,2,3,4,5\}$, then accept. \\
\indent \indent \indent 2. If $s = w$, then run $M$ on $w$ and output what $M$ outputs.' \\

\indent 2. Output $\langle M_w \rangle$". \\

\setlength\parindent{0pt}

And if $G$ is given a string that is not an encoding of a TM $M$ and a string $w$, then output $\langle M_5 \rangle$ where $M_5$ is the TM that accepts exactly the first 5 strings of $\Sigma^*$ in standard string order. \\

Hence, 
\[
  g(x) =
  \begin{cases}
        \langle M_w \rangle & \text{if $x = \langle M, w \rangle$ for some TM $M$ and string $w$} \\
        \langle M_5 \rangle & \text{otherwise} 
  \end{cases}
\]

We have that $\langle M, w \rangle \in A_{TM}$ if and only if $f(\langle M, w \rangle) \in \overline{A}$. \\

So we have a mapping reduction $g$ from $A_{TM}$ to $\overline{A}$. Hence, $A_{TM} \leq_m \overline{A}$. \\

Since $A_{TM} \leq_m \overline{A}$, we also have $\overline{A_{TM}} \leq_m \overline{\overline{A}}$. Since $\overline{\overline{A}} = A$, we have that $\overline{A_{TM}} \leq_m A$. Since we know that $\overline_{A_{TM}}$ is not recognizable from the textbook, we have that $A$ is not recognizable. \\

\textbf{Conclusion:} $A$ is not Turing-recognizable and $A$ is not Turing co-recognizable. 

\subsection*{Part b)}

\textbf{Required:} Let $B = \{\langle M \rangle : \text{$M$ is a TM and } |L(M)| \geq 5\}$. Is $A$ Turing recognizable? Is $A$ Turing co-recognizable. \\

We will show that $B$ is Turing recognizable, but not Turing co-recognizable. \\

\textbf{Def:} An ordering of strings is in Standard String Order (SSO) if they are first ordered by length, and strings of the same length are ordered by some alphabetical ordering. \\

\textbf{Note:} Standard String Order (SSO) is also known as lexicographical ordering. In Tutorials, we used the term Standard String Order (SSO). \\

First consider an enumeration of $\Sigma^*$ in Standard String Order (SSO). \\

Consider the following TM $N$ that recognizes $B$. \\

$N = $ "On input $\langle M \rangle$: \\

\setlength\parindent{15pt}

\indent 1. For each $w \in \Sigma^*$ in standard string order: simulate $M$ on $w$. If $M$ accepts $w$, keep track of this result. \\

\indent 2. If $w$ has accepted 5 distinct strings $w$, then accept." \\

\setlength\parindent{0pt}
And if $N$ is given an input that is not an encoding of a TM $M$, then simply reject. \\

To show that $N$ is a recognizer of $B$, let $x \in B$. Hence, $x = \langle M \rangle$ such that $|L(M)| \geq 5$. Hence, $N$ will simulate $M$ on $w$ for each $w \in \Sigma^*$ in standard string order. Since $|L(M)| \geq 5$, eventually $M$ will accept 5 strings, and hence $N$ will accept $x$. \\

If $x \not \in B$ and $x$ is not an encoding of a TM, then $N$ will not accept $x$. If $x \not \in B$ and $x = \langle M \rangle$ for some TM $M$, then $M$ accepts at most 4 strings. Hence, $N$ will not accept $x$. \\

Therefore, $N$ recognizes $B$. Therefore, $B$ is Turing-recognizable. \\

Now we will show that $B$ is not Turing co-recognizable. In lecture we proved Rice's Theorem. \\

\textbf{Rice's Theorem:} Suppose $P$ is a language of TM descriptions such that \\

(i) $P$ is non-trivial: it contains some but not all TM descriptions. \\

(ii) If $L(M_1) = L(M_2)$ for TMs $M_1$ and $M_2$, then $\langle M_1 \rangle \in P$ if and only if $\langle M_2 \rangle \in P$. \\

Then $P$ is not decidable. \\

Clearly $B$ is a language of TM descriptions. And clearly $B$ does not contain all TM descriptions. For instance $\langle M_{reject} \rangle \not \in B$ where $M_{reject}$ is the TM that rejects all strings since $|L(M_{reject})| = 0$. And clearly $B$ contains some TM descriptions. For instance, $\langle M_5 \rangle \in B$, where $M_5$ is the TM that accepts exactly the first 5 strings of $\Sigma^*$ in standard string order. \\

And, if $L(M_1) = L(M_2)$ for TMs $M_1$ and $M_2$, then $|L(M_1)| = |L(M_2)|$. Hence, trivially we have $\langle M_1 \rangle \in B$ if and only if $\langle M_2 \rangle \in B$. \\

Hence, $B$ is not decidable by Rice's Theorem. \\

By Theorem 4.22 in the textbook, we know that a language is decidable if and only if it is recognizable and co-recognizable. \\

Hence, a language is not decidable if and only if it is not recognizable or not co-recognizable. \\

Since $B$ is not decidable, we know $B$ is not recognizable or $B$ is not co-recognizable. \\

Since we have shown that $B$ is recognizable, we must have that $B$ is not co-recognizable. \\

\textbf{Conclusion:} $B$ is Turing-recognizable, but $B$ is NOT Turing co-recognizable.

\newpage

\section*{Question 3}

\textbf{Required:} Show that for every language $A$, there exists a language $B$ such that $B \not \leq_m A$. 

\begin{proof}
Assume for the sake of contradiction that there exists a language $A$ such that for every language $B$, we have that $B \leq_m A$. \\

Let $\Sigma$ be the finite alphabet that we are working with. \\

We know that every language is a subset of $\Sigma^*$ which is an infinite set. \\

Let $\Sigma_{i}$ be the set of all strings that are of length $i \in \N$. Hence, $\Sigma^* = \bigcup_{i = 0}^\infty \Sigma_i$. Since each $\Sigma_i$ is finite, each $\Sigma_i$ is countable. Hence, $\Sigma^* = \bigcup_{i = 0}^\infty \Sigma_i$ is a countable union of countable sets. Hence, $\Sigma^*$ is countably infinite. \\

Since $\Sigma^*$ is countably infinite, we know that its powerset, $\mathcal{P}(\Sigma^*)$ is uncountable. \\

We also know from lecture that there are only countably many Turing machines. Since there are only countably many Turing machines, there are only countably many computable functions. \\

We assumed every $B \in P(\Sigma^*)$ is such that $B \leq_m A$. Since there are uncountably many sets $B \in P(\Sigma^*)$, and there are only countably many computable functions, there must exist at least two distinct sets $B_1, B_2 \in P(\Sigma^*)$ such that $B_1 \neq B_2$ where the mapping reductions for $B_1 \leq_m A$ and $B_2 \leq_m A$ are given by the same computable function $f: \Sigma^* \rightarrow \Sigma^*$. \\

Hence, for all $w \in \Sigma^*$, we have the following by definition of mapping reducibility for $B_1 \leq_m A$ and $B_2 \leq_m A$. 
\begin{align}
w \in B_1 \Leftrightarrow f(w) \in A
\end{align}
\begin{align}
w \in B_2 \Leftrightarrow f(w) \in A
\end{align}

Since $B_1 \neq B_2$, there exists a string in one of these sets, but not the other. Without loss of generality, assume $s \in B_1$ but $s \not \in B_2$. \\

Since $s \in B_1$, by (1) we get that $f(s) \in A$. \\

Since $s \not \in B_2$, by (2) we get that $f(s) \not \in A$. \\

Notice, $f(s) \in A$ and $f(s) \not \in A$ is a contradiction. Therefore, our initial assumption was wrong. \\

Therefore, for every language $A$, there exists a language $B$ such that $B \not \leq_m A$. This completes the proof, as required. 
\end{proof}

\newpage

\section*{Question 4}

\textbf{Required:} A useless state in a Turing machine is one that is never entered on any input string. Consider the problem of determining whether a Turing machine has any useless states. Formulate this problem as a language and show that is it undecidable. 

\begin{proof}
Let $A = \{\langle M \rangle | \text{$M$ is a TM that has NO useless states}\}$ \\

\textbf{NOTE:} The textbook defines every Turing machine $M$ to have an accept state $q_{accept}$ and a reject state $q_{reject}$ such that $q_{accept} \neq q_{reject}$. This is given on page 168, definition 3.3. 

From the textbook we know that $HALT$ is undecidable, where $HALT$ is defined as follows. 

$$HALT = \{\langle M, w \rangle : \text{$M$ is a TM and $M$ halts on $w$}\}$$

We will show that $HALT \leq_m A$ by defining a mapping reduction $f$ given by the following TM $F$. \\

$F = $ "On input $\langle M, w \rangle$: \\

\setlength\parindent{15pt}

\indent 1. Construct the following TM $M_w$ with no useless states except possibly $q_{accept}$. \\
\indent \indent $M_w = $ 'On input $s$: \\
\indent \indent \indent 1. If $s \neq w$, then reject. \\
\indent \indent \indent 2. If $s = w$, run $M$ on $w$. If $M$ halts on $w$, then accept.' \\

\indent 2. Output $\langle M_w \rangle$" \\

\setlength\parindent{0pt}

And if $F$ is given a string $x$ such that $x \neq \langle M, w \rangle$ for some TM $M$ and string $w$, then output $\langle M_{reject} \rangle$ where $M_{reject}$ is the TM that rejects all inputs (and hence $q_{accept}$ is useless). \\

So $F$ determines the following computable function $f: \Sigma^* \rightarrow \Sigma^*$. 
\[
  f(x) =
  \begin{cases}
        \langle M_w \rangle & \text{if $x = \langle M, w \rangle$ for some TM $M$ and string $w$} \\
        \langle M_{reject} \rangle & \text{otherwise} 
  \end{cases}
\]
And $\langle M, w \rangle \in HALT$ if and only if $f(\langle M, w \rangle)$ has no useless states. \\

Therefore, $\langle M, w \rangle \in HALT$ if and only if $f(\langle M, w \rangle) \in A$. \\

Hence, $f$ is a mapping reduction from $HALT$ to $A$. i.e. $HALT \leq_m A$. \\

We know $HALT$ and $\overline{HALT}$ are both undecidable from lecture and the textbook. Since $HALT \leq_m A$ and $HALT$ is undecidable, we have that $A$ is undecidable. Since $HALT \leq_m A$, we also have $\overline{HALT} \leq_m \overline{A}$. Since $\overline{HALT} \leq_m \overline{A}$ and $\overline{HALT}$ is undecidable, we also have that $\overline{A}$ is undecidable. \\

Therefore, since $A$ and $\overline{A}$ are undecidable, we have shown that the problem of determining whether a TM has any useless states is undecidable. 
\end{proof}










\end{document} 




